\documentclass[12pt]{article}
\usepackage[russian]{babel}
\usepackage{amsmath}
\usepackage{amssymb}
\usepackage{fontspec}
\usepackage[T1,T2A]{fontenc}
\usepackage{graphicx}
\usepackage{listings}
\setmainfont{Liberation Serif}
\oddsidemargin = -5mm
\topmargin = -20mm
\textheight = 230mm
\textwidth = 170mm
\usepackage{indentfirst}
\usepackage{xcolor}


\begin{document}

{
	\centerline{Суперкомпьютерное моделирование и технологии}
	\medskip
	\centerline{Задание 2}
	{
		\bfseries
		\large
		\centerline{Численное интегрирование многомерных функций}
		\centerline{методом Монте-Карло}
	}
	\medskip
	\rightline{Кузьмин Ярослав Константинович}
	\rightline{621 группа}
	\rightline{14 вариант}
	\medskip
}

\section{Постановка задачи}

Дана функция $f(x, y, z) = \sin(x^2 + z^2) y$.
Требуется написать программу с использованием технологии MPI для
вычисления значения интеграла методом Монте-Карло.

$$
I = \int\int\limits_G\int f(x, y, z) dx dy dz
$$

$$
G = \{(x, y, z) : x \geqslant 0, y \geqslant 0, z \geqslant 0, x^2 + y^2 +
z^2 \leqslant 1\}
$$

Проведём аналитическое решение данного интеграла.
Перейдём к цилиндрической системе координат:

{
	$$\left\{
	\begin{tabular}{l}
		$y = y$ \\
		$x = r \cos\phi$ \\
		$z = r \sin\phi$ \\
		$J = r$ \\
	\end{tabular}
	\right.$$
}

Интеграл принимает вид
$$
I = \int\limits_0^1 \int\limits_0^{\frac{\pi}{2}} \int\limits_0^{\sqrt{1 - y^2}}
\sin(r^2) y r dr d\phi dy =
\frac{1}{2}
\int\limits_0^1 y \int\limits_0^{\frac{\pi}{2}} \int\limits_0^{\sqrt{1 - y^2}}
\sin(r^2) d(r^2) d\phi dy =
$$
$$
= - \frac{1}{2}
\int\limits_0^1 y \int\limits_0^{\frac{\pi}{2}}
\left. \cos(r^2) \right|_0^{\sqrt{1 - y^2}} d\phi dy =
- \frac{1}{2}
\int\limits_0^1 y \int\limits_0^{\frac{\pi}{2}}
(\cos(1 - y^2) - 1) d\phi dy =
$$
$$
= \frac{1}{2}
\int\limits_0^1 y \int\limits_0^{\frac{\pi}{2}} d\phi dy -
\frac{1}{2}
\int\limits_0^1 y \int\limits_0^{\frac{\pi}{2}}
\cos(1 - y^2) d\phi dy =
\frac{\pi}{4}
\int\limits_0^1 y dy -
\frac{\pi}{4}
\int\limits_0^1 \cos(1 - y^2) y dy =
$$
$$
= \frac{\pi}{8} \left. y^2 \right|_0^1 +
\frac{\pi}{8}
\int\limits_0^1 \cos(1 - y^2) d(1 - y^2) =
\frac{\pi}{8} +
\frac{\pi}{8} \left. \sin(1 - y^2) \right|_0^1 =
$$
$$
= \frac{\pi}{8} (1 - \sin(1)) \approx 0.06225419868854213
$$

\section{Поиск приближённого решения}

Поиск значения интеграла методом Монте-Карло работает следующим образом:
рассматривается прямоугольный параллелепипед
$\Pi = \{(x, y, z) :
a_1 \leqslant x \leqslant b_1,
a_2 \leqslant y \leqslant b_2,
a_3 \leqslant z \leqslant b_3\}$ такой что
$G \subseteq \Pi$.

$$
F(x, y, z) = \begin{cases}
	f(x, y, z),& (x, y, z) \in G \\
	0,& (x, y, z) \notin G \end{cases}
$$

Случайным образом в $\Pi$ выбираются $n$ точек $p_1, \ldots, p_n$.
Приближённое значение интеграла вычисляется по формуле
$$
I \approx |\Pi|\frac{1}{n}\sum_{i=1}^n F(p_i)
$$
$$
|\Pi| = (b_1 - a_1)(b_2 - a_2)(b_3 - a_3)
$$

\section{Описание программы}

Программа использует парадигму <<мастер --- рабочие>>.
Процесс с индексом 0 генерирует точки и распределяет их для вычисления суммы
между остальными процессами с помощью функции MPI\_Scatterv.
Остальные процессы при получении своего сегмента
точек вычисляют значение $F(p)$ в них и суммируют полученые значения.
Далее происходит поиск суммы значений, полученных процессами,
с помощью операции редукции с использованием функции MPI\_Reduce.
Результат редукции получает процесс с индексом 0, который вычисляет
приближённое значение интеграла и принимает решение о продолжении вычислений
или об их остановке на основе значения интеграла, полученного аналитически и
значения $\varepsilon$. На каждой итерации генерируется 20000 точек.

\section{Результаты}

В ходе запусков на системе Polus были получены следующие результаты.

\begin{center}
	\begin{tabular}{|l|l|l|l|l|}
		\hline
		Точность $\varepsilon$ &
		Число MPI-процессов &
		Время работы программы (с) &
		Ускорение &
		Ошибка \\
		\hline
		& 2 & 0.061308 & 1 & $1.83 \cdot 10^{-5}$ \\
		\cline{2-5}
		$3.0 \cdot 10^{-5}$
		& 4 & 0.2352 & 0.260663 & $1.553 \cdot 10^{-5}$ \\
		\cline{2-5}
		& 16 & 0.196815 & 0.311501 & $4.8 \cdot 10^{-6}$ \\

		\hline
		& 2 & 0.927436 & 1 & $1.29 \cdot 10^{-6}$ \\
		\cline{2-5}
		$5.0 \cdot 10^{-6}$
		& 4 & 0.442445 & 2.096161 & $2.35 \cdot 10^{-6}$ \\
		\cline{2-5}
		& 16 & 0.664359 & 1.395986 & $3.02 \cdot 10^{-6}$ \\

		\hline
		& 2 & 1.042007 & 1 & $1.4 \cdot 10^{-6}$ \\
		\cline{2-5}
		$1.5 \cdot 10^{-6}$
		& 4 & 0.327713 & 3.179632 & $4.36 \cdot 10^{-7}$ \\
		\cline{2-5}
		& 16 & 2.791492 & 0.373279 & $4.82 \cdot 10^{-7}$ \\
		\hline
	\end{tabular}

	\medskip

	\includegraphics[width=0.9\textwidth]{plot.png}
\end{center}

В ходе запусков было установлено, что генерация точек в одном процессе
явояется узким местом. Для ускорения генерации точек содержащий
её цикл был распараллелен с помощью OpenMP. Из графика видно, что при
большой погрешности использование большего числа процессов приводит к
уменьшению скорости работы. Это может быть связано с тем, что накладные
расходы от использования множества процессов превосходят ускорение.
Однако при увеличении точности наблюдается увеличение скорости работы
при увеличении количества процессов. Однако при дальнейшем увеличении
количества процессов скорость работы начинает уменьшаться из-за увеличения
накладных расходов.

\end{document}
